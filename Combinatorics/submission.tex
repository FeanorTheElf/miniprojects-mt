\documentclass{scrartcl}

\usepackage{graphicx}
\usepackage[utf8]{inputenc}
\usepackage[T1]{fontenc}
\usepackage[english]{babel}
\usepackage{amsmath}
\usepackage{mathtools}
\usepackage{amssymb}
\usepackage{amsthm}
\usepackage{listings}
\usepackage[style=english]{csquotes}
\usepackage[language=english, backend=biber, style=alphabetic, sorting=nyt]{biblatex}

\addbibresource{bibliography.bib}

\title{Miniproject - Combinatorics}
\author{Simon Pohmann}
\date{}

\newcommand{\R}{\mathbb{R}}
\newcommand{\N}{\mathbb{N}}
\newcommand{\Z}{\mathbb{Z}}
\newcommand{\Q}{\mathbb{Q}}
\newcommand{\C}{\mathbb{C}}

\newcommand\restr[2]{{
    \left.\kern-\nulldelimiterspace
    #1
    \vphantom{\big|}
    \right|_{#2}
}}

\theoremstyle{definition}
\newtheorem{definition}{Definition}
\newtheorem{lemma}[definition]{Lemma}
\newtheorem{remark}[definition]{Remark}
\newtheorem{proposition}[definition]{Proposition}
\newtheorem{example}[definition]{Example}
\newtheorem{corollary}[definition]{Corollary}

\begin{document}
\maketitle

We use the convention that $\N = \{ n \in \Z \ | \ n \geq 0 \}$.

\section{Part I}

\begin{proposition}
    \label{prop:definition_rank_function}
    Let $P$ be a graded poset in which every maximal chain has length $n + 1$. Then the function
    \begin{equation*}
        r: P \to \{ 0, ..., n \}, \quad x \mapsto \max\{ k \in \N \ | \ \exists a_1, ..., a_k \in P: \ a_1 < a_2 < ... < a_k < x \}
    \end{equation*}
    is well-defined and the unique function with $x < y$ implies $r(x) < r(y)$ for all $x, y \in P$.
    We call it the \emph{rank function} of $P$.
\end{proposition}
\begin{proof}
    Clearly $r$ is well-defined, as for every increasing sequence $a_1 < ... < a_k < x$, we have the chain $C := \{ a_1, ..., a_k, x \}$ of size $k + 1$.
    Hence by assumption, $k \leq n$ and so $r(x)$ is finite and in $\{ 0, ..., n \}$.

    Further, consider $x < y$ in $P$.
    We have a sequence $a_1 < ... < a_{r(x)} < x$ by definition of $r$.
    It follows that there is an increasing sequence $a_1 < ... < a_{r(x)} < x < y$ and so $r(y) \geq r(k) + 1 > r(x)$.

    Lastly, assume there was another function $r': P \to \{ 0, ..., n \}$ with this property.
    Consider any $x \in P$.
    By definition of $r$, there is an increasing sequence $a_1 < ... < a_{r(x)} < x$ in $P$.
    Now consider a maximal chain $C$ containing the chain $\{ a_1, ..., a_{r(x)}, x \}$.

    Say $C = \{ b_1, ..., b_{n + 1} \}$ with $b_1 < ... < b_{n + 1}$ and $x = b_j$.
    Note that we have the increasing sequence $b_1 < ... < b_{j - 1} < b_j = x$ and so by the definition of $r$, find $j \leq r(x) + 1$.
    On the other hand, have $a_1, ..., a_{r(x)} \in C$ and thus $j = r(x) + 1$, i.e.
    \begin{equation*}
        a_1 = b_1, \ ..., \ a_{r(x)} = b_{r(x)}, \ x = b_{r(x) + 1}
    \end{equation*}
    As $b_i < b_{i + 1}$, we know that $r'(b_i) < r'(b_{i + 1})$ and inductively, we see $r'(b_i) \geq i - 1$.
    However, $r'(b_{n + 1}) \leq n$ and thus $r'(b_i) = i - 1$.
    Finally it follows that $r'(x) = r'(b_{r(x) + 1}) = r(x)$.
\end{proof}
Now we will show some basic properties of the rank function.
\begin{proposition}
    Let $P$ be a graded poset of maximal rank $n$ with rank function $r$.
    Then
    \begin{itemize}
        \item $x \in P$ is minimal iff $r(x) = 0$ and maximal iff $r(x) = n$.
        \item If $x < y$ and $r(x) + 1 \neq r(y)$ then there is $z \in P$ with $x < z < y$.
        \item If $x < y$ then there is an increasing sequence $x < a_1 < ... < a_{r(y) - r(x) - 1} < y$ in $P$. 
    \end{itemize}
\end{proposition}
\begin{proof}
    For (i), let $x \in P$ be minimal. Then there is no increasing sequence $a_1 < x$ in $P$, so $r(x) \leq 0$.
    Conversely, let $r(x) = 0$. Assume there was $y \in P$ with $y < x$, then $r(y) < r(x) = 0$, a contradiction.
    The analogous statement for maximal elements is proved in the same way.
    
    For (iii), consider $x < y$ in $P$. Then the chain $\{ x, y \}$ is contained in a maximal chain $C \subseteq P$.
    Say $C = \{ b_1, ..., b_{n + 1} \}$ with $b_1 < ... < b_{n + 1}$.

    Hence we find $r(b_i) < r(b_{i + 1})$ and so inductively that $r(b_i) \geq i - 1$ and $r(b_i) \leq i - 1$ since $r(b_1) \geq 0$ and $r(b_{n + 1}) \leq n$.
    It follows that $r(b_i) = i - 1$ and so $x = b_{r(x) + 1}, y = {r(y) + 1}$.
    Therefore, we have a chain
    \begin{equation*}
        x < b_{r(x) + 2} < ... < b_{r(y)} < y
    \end{equation*}
    of length $r(y) - r(x) - 1$.
    Statement (ii) follows directly, as in this setting, have $r(y) \geq r(x) + 2$ and so $r(y) - r(x) - 1 \neq 0$.
\end{proof}

\section{Part II}
\begin{proposition}
    For a graded poset $P$ with layers $L_0, ..., L_n$ the following statements are equivalent:
    \begin{itemize}
        \item For every antichain $A \subseteq P$ have
        \begin{equation*}
            \sum_{i = 0}^n \frac {|A \cap L_i|} {|L_i|} \leq 1
        \end{equation*}
        \item For each $1 \leq i \leq n$ and $F \subseteq L_i$ have
        \begin{equation*}
            \frac {|\partial F|} {|L_{i - 1}|} \geq \frac {|F|} {|L_i|}
        \end{equation*}
        where
        \begin{equation*}
            \partial F := \{ a \in L_{i - 1} \ | \ \exists b \in F: \ a \leq b \}
        \end{equation*}
        \item There exists a (nonempty) sequence of maximal chains $C_1, ..., C_t$ such that for all $1 \leq i \leq n$ we have:
        \begin{equation*}
            |\{ j \ | \ x \in C_j \}| = |\{ j \ | \ y \in C_j \}| \ \text{for all $x, y \in L_i$}
        \end{equation*}
    \end{itemize}
\end{proposition}
\begin{proof}
    We show (ii) $\Rightarrow$ (i) $\Rightarrow$ (iii) $\Rightarrow$ (ii).
    For convenience of notation, write
    \begin{equation*}
        A_i := \bigcup_{i \leq j \leq n} A \cap L_j
    \end{equation*}
    
    \paragraph{(ii) $\Rightarrow$ (i)} Define sets
    \begin{equation*}
        G_n := A \cap L_n \quad \text{and} \quad G_i := \partial G_{i + 1} \cup (A \cap L_i) \ \text{for $0 \leq i < n$}
    \end{equation*}
    We show by induction that for $a \in G_i$ there is some $b \in A_i$ with $a \leq b$ and that we have the inequality
    \begin{equation*}
        \sum_{j = i}^n \frac {|A \cap L_j|} {|L_j|} \leq \frac {|G_i|} {|L_i|}
    \end{equation*}
    The base case is trivial, so let $i < n$.
    Consider some $a \in G_i$.
    If $a \in \partial G_{i + 1}$, then there is $b \in G_{i + 1}$ with $a \leq b$.
    By induction hypothesis, have $c \in A_{i + 1} \supseteq A_i$ with $b \leq c$ and thus $a \leq c$.
    Otherwise, find $a \in A \cap L_i$ and so $a \leq a$ with $a \in A \cap L_i \subseteq A_i$.

    To show the inequality, note that $\partial G_{i + 1}$ and $A \cap L_i$ are disjoint.
    Indeed, if $a \in \partial G_{i + 1} \cap A \cap L_i$ then there is $b \in G_{i + 1}$ with $a \leq b$, and further by induction hypothesis there is $c \in A_{i + 1}$ with $a \leq b \leq c$.
    However, $a \in L_i$ and so $a \neq c$.
    So we found $a \leq c$ comparable elements in the antichain $A$, a contradiction.

    So we get
    \begin{align*}
        \sum_{j = i}^n \frac {|A \cap L_j|} {|L_j|} =& \frac {|A \cap L_i|} {|L_i|} + \sum_{j = i + 1}^n \frac {|A \cap L_j|} {|L_j|} \\
        \leq& \frac {|A \cap L_i|} {|L_i|} + \frac {|G_{i + 1}|} {|L_{i + 1}|} \leq \frac {|A \cap L_i|} {|L_i|} + \frac {|\partial G_{i + 1}|} {|L_i|} \\
        =& \frac {|A \cap L_i| + |\partial G_{i + 1}|} {|L_i|} = \frac {|G_i|} {|L_i|}
    \end{align*}
    Finally, we have that $G_0 \subseteq L_0$, so
    \begin{equation*}
        \sum_{j = 0}^n \frac {|A \cap L_j|} {|L_j|} \leq \frac {|G_i|} {|L_i|} \leq 1
    \end{equation*}

    \paragraph{(i) $\Rightarrow$ (iii)}
\end{proof}

\end{document}