\documentclass{scrartcl}

\usepackage{graphicx}
\usepackage[utf8]{inputenc}
\usepackage[T1]{fontenc}
\usepackage[english]{babel}
\usepackage{amsmath}
\usepackage{mathtools}
\usepackage{amssymb}
\usepackage{amsthm}
\usepackage{listings}
\usepackage[style=english]{csquotes}
\usepackage[language=english, backend=biber, style=alphabetic, sorting=nyt]{biblatex}

\addbibresource{bibliography.bib}

\title{Miniproject - Combinatorics}
\author{Simon Pohmann}
\date{}

\newcommand{\R}{\mathbb{R}}
\newcommand{\N}{\mathbb{N}}
\newcommand{\Z}{\mathbb{Z}}
\newcommand{\Q}{\mathbb{Q}}
\newcommand{\C}{\mathbb{C}}

\newcommand\restr[2]{{
    \left.\kern-\nulldelimiterspace
    #1
    \vphantom{\big|}
    \right|_{#2}
}}

\theoremstyle{definition}
\newtheorem{definition}{Definition}
\newtheorem{lemma}[definition]{Lemma}
\newtheorem{remark}[definition]{Remark}
\newtheorem{proposition}[definition]{Proposition}
\newtheorem{example}[definition]{Example}
\newtheorem{corollary}[definition]{Corollary}

\begin{document}
\maketitle

We use the convention that $\N = \{ n \in \Z \ | \ n \geq 0 \}$.

\section{Part I}

\begin{proposition}
    \label{prop:definition_rank_function}
    Let $P$ be a graded poset in which every maximal chain has length $n + 1$. Then the function
    \begin{equation*}
        r: P \to \{ 0, ..., n \}, \quad x \mapsto \max\{ k \in \N \ | \ \exists a_1, ..., a_k \in P: \ a_1 < a_2 < ... < a_k < x \}
    \end{equation*}
    is well-defined and the unique function with $x < y$ implies $r(x) < r(y)$ for all $x, y \in P$.
    We call it the \emph{rank function} of $P$.
\end{proposition}
\begin{proof}
    Clearly $r$ is well-defined, as for every increasing sequence $a_1 < ... < a_k < x$, we have the chain $C := \{ a_1, ..., a_k, x \}$ of size $k + 1$.
    Hence by assumption, $k \leq n$ and so $r(x)$ is finite and in $\{ 0, ..., n \}$.

    Further, consider $x < y$ in $P$.
    We have a sequence $a_1 < ... < a_{r(x)} < x$ by definition of $r$.
    It follows that there is an increasing sequence $a_1 < ... < a_{r(x)} < x < y$ and so $r(y) \geq r(k) + 1 > r(x)$.

    Lastly, assume there was another function $r': P \to \{ 0, ..., n \}$ with this property.
    Consider any $x \in P$.
    By definition of $r$, there is an increasing sequence $a_1 < ... < a_{r(x)} < x$ in $P$.
    Now consider a maximal chain $C$ containing the chain $\{ a_1, ..., a_{r(x)}, x \}$.

    Say $C = \{ b_1, ..., b_{n + 1} \}$ with $b_1 < ... < b_{n + 1}$ and $x = b_j$.
    Note that we have the increasing sequence $b_1 < ... < b_{j - 1} < b_j = x$ and so by the definition of $r$, find $j \leq r(x) + 1$.
    On the other hand, have $a_1, ..., a_{r(x)} \in C$ and thus $j = r(x) + 1$, i.e.
    \begin{equation*}
        a_1 = b_1, \ ..., \ a_{r(x)} = b_{r(x)}, \ x = b_{r(x) + 1}
    \end{equation*}
    As $b_i < b_{i + 1}$, we know that $r'(b_i) < r'(b_{i + 1})$ and inductively, we see $r'(b_i) \geq i - 1$.
    However, $r'(b_{n + 1}) \leq n$ and thus $r'(b_i) = i - 1$.
    Finally it follows that $r'(x) = r'(b_{r(x) + 1}) = r(x)$.
\end{proof}
Now we will show some basic properties of the rank function.
\begin{proposition}
    Let $P$ be a graded poset of maximal rank $n$ with rank function $r$.
    Then
    \begin{itemize}
        \item $x \in P$ is minimal iff $r(x) = 0$ and maximal iff $r(x) = n$.
        \item If $x < y$ and $r(x) + 1 \neq r(y)$ then there is $z \in P$ with $x < z < y$.
        \item If $x < y$ then there is an increasing sequence $x < a_1 < ... < a_{r(y) - r(x) - 1} < y$ in $P$. 
    \end{itemize}
\end{proposition}
\begin{proof}
    For (i), let $x \in P$ be minimal. Then there is no increasing sequence $a_1 < x$ in $P$, so $r(x) \leq 0$.
    Conversely, let $r(x) = 0$. Assume there was $y \in P$ with $y < x$, then $r(y) < r(x) = 0$, a contradiction.
    The analogous statement for maximal elements is proved in the same way.
    
    For (iii), consider $x < y$ in $P$. Then the chain $\{ x, y \}$ is contained in a maximal chain $C \subseteq P$.
    Say $C = \{ b_1, ..., b_{n + 1} \}$ with $b_1 < ... < b_{n + 1}$.

    Hence we find $r(b_i) < r(b_{i + 1})$ and so inductively that $r(b_i) \geq i - 1$ and $r(b_i) \leq i - 1$ since $r(b_1) \geq 0$ and $r(b_{n + 1}) \leq n$.
    It follows that $r(b_i) = i - 1$ and so $x = b_{r(x) + 1}, y = {r(y) + 1}$.
    Therefore, we have a chain
    \begin{equation*}
        x < b_{r(x) + 2} < ... < b_{r(y)} < y
    \end{equation*}
    of length $r(y) - r(x) - 1$.
    Statement (ii) follows directly, as in this setting, have $r(y) \geq r(x) + 2$ and so $r(y) - r(x) - 1 \neq 0$.
\end{proof}

\section{Part II}
\begin{proposition}
    For a graded poset $P$ with layers $L_0, ..., L_n$ the following statements are equivalent:
    \begin{itemize}
        \item For every antichain $A \subseteq P$ have
        \begin{equation*}
            \sum_{i = 0}^n \frac {|A \cap L_i|} {|L_i|} \leq 1
        \end{equation*}
        \item For each $1 < i \leq n$ and $F \subseteq L_i$ have
        \begin{equation*}
            \frac {|\partial F|} {|L_{i - 1}|} \geq \frac {|F|} {|L_i|}
        \end{equation*}
        where
        \begin{equation*}
            \partial F := \{ a \in L_{i - 1} \ | \ \exists b \in F: \ a \leq b \}
        \end{equation*}
        \item There exists a (nonempty) sequence of maximal chains $C_1, ..., C_t$ such that for all $1 \leq i \leq n$ we have:
        \begin{equation*}
            |\{ j \ | \ x \in C_j \}| = |\{ j \ | \ y \in C_j \}| \ \text{for all $x, y \in L_i$}
        \end{equation*}
    \end{itemize}
\end{proposition}
\begin{proof}
    Show the directions (ii) $\Rightarrow$ (iii) $\Rightarrow$ (i) $\Rightarrow$ (ii)
    \paragraph{(ii) $\Rightarrow$ (iii)} Use induction on $n$.
    Again, the base case is trivial, just take chains $\{x\}$ for each $x \in A = L_0$.
    So assume that $n > 0$. The set
    \begin{equation*}
        A' := \bigcup_{i = 0}^{n - 1} L_i
    \end{equation*}
    is a graded poset of maximal rank $n - 1$, and hence there are maximal chains $C_1, ..., C_t \subseteq A'$ such that for all $1 \leq i \leq n - 1$ have
    \begin{equation*}
        |\{j \ | \ x \in C_j\}| = |\{j \ | \ y \in C_j\}| \ \text{for all $x, y \in L_i$}
    \end{equation*}
    by induction hypothesis.
    Let now $V_1, V_2 := [t] \times L_n$ (treat them as disjoint) and consider the bipartite graph $G := (V_1 \sqcup V_2, E)$ where $E$ is defined as follows:
    \begin{equation*}
        \{ \underbrace{(i, a)}_{\in V_1}, \underbrace{(j, b)}_{\in V_2} \} \in E \ \Leftrightarrow \ \max C_j < a
    \end{equation*}
    We use Hall's theorem to show that $G$ has a perfect matching.

    Consider any $W = \{ (i_1, a_1), ..., (i_w, a_w) \} \subseteq V_1$ and let $F = \{ a_1, ..., a_w \}$.
    Then
    \begin{equation*}
        W \subseteq [t] \times F \ \Rightarrow \ |W| \leq t|F|
    \end{equation*}
    By choice of $C_1, ..., C_t$, we have that the number of $j$ with $x \in C_j$ is the same for all $x \in L_{n - 1}$, say $k$.
    Since the $C_i$ are maximal chains, each intersects $L_{n - 1}$ in exactly one element.
    So have bijection
    \begin{align*}
        \{ (x, i) \ | \ x \in L_{n - 1}, 1 \leq i \leq t, x \in C_i\} \to \{ i \ | \ 1 \leq i \leq t \}, \quad (x, i) \mapsto i
    \end{align*}
    where the set on the left-hand side has size $k |L_{n - 1}|$. It follows that $k|L_{n - 1}| = t$.

    Using this, we find
    \begin{align*}
        N(W) =& \{ (j, b) \ | \ b \in L_n, \exists a \in F: \max C_j < a \} \\
        =& L_n \times \{ j \ | \ \exists a \in F: \max C_j = a \} \\
        =& L_n \times \{ j \ | \ \max C_j \in \partial F \}
    \end{align*}
    and so
    \begin{align*}
        |N(W)| = |L_n| \sum_{\max C_j \in \partial F} 1 = |L_n| \sum_{x \in \partial F} \sum_{x \in C_j} 1 = |L_n| \sum_{x \in \partial F} k = |L_n| |\partial F| k
    \end{align*}
    Together with the assumption that $|\partial F| |L_n| \geq |F| |L_{n - 1}|$ we see that
    \begin{equation*}
        |W| \leq t |F| = k |L_{n - 1}| |F| \leq k |\partial F| |L_n| = |N(W)|
    \end{equation*}
    So Hall's theorem yields a perfect matching $M \subseteq E$ from $V_1$ to $V_2$.
    As $|V_1| = |V_2|$, this is already a 1-to-one correspondence.

    Now consider the sets
    \begin{equation*}
        C'_m := C_j \cup \{ a \} \ \text{where} \ m = \{\underbrace{(i, a)}_{\in V_1}, \underbrace{(j, b)}_{\in V_2}\} \in M
    \end{equation*}
    These are chains, as $\max C_j < a$ for each $C'_m$.
    Further, for each $x, y \in L_i, i < n$ have that
    \begin{equation*}
        |\{ C'_m \ | \ x \in C'_m \}| = |L_n \times \{ j \ | \ x \in C_j \}| = |L_n \times \{ j \ | \ y \in C_j \}| = |\{ C'_m \ | \ y \in C'_m \}|
    \end{equation*}
    as $M$ is a matching from $V_2$ to $V_1$.
    Finally, for all $x \in L_n$ we have that
    \begin{align*}
        |\{ C'_m \ | \ x \in C'_m \}| =& |\{ m \in M \ | \ \exists i, v \in V_2: m = \{ (i, x), v \} \}| = |\{ (i, x) \ | \ (i, x) \in V_1 \}| = t
    \end{align*}
    as $M$ is a matching from $V_1$ to $V_2$.

    \paragraph{(iii) $\Rightarrow$ (i)} Let $C_1, ..., C_t$ be a sequence of maximal chains given by the assumption.
    For $1 \leq i \leq n$ let $k_i$ be the number of different $j$ such that a fixed element $x \in L_i$ is contained in exactly the $C_j$.
    By assumption, this does not depend on $x$.

    As in the direction above, we see that $k_i |L_i| = t$ because each $C_j$ intersects $L_i$ in exactly one element, so there is a bijection
    \begin{align*}
        \{ (x, j) \ | \ x \in L_i, 1 \leq j \leq t, x \in C_j\} \to \{ j \ | \ 1 \leq j \leq t \}, \quad (x, j) \mapsto j
    \end{align*}
    and the set on the left has size $k_i |L_i|$.

    Since each $C_j$ is a chain and $A$ is an antichain, we find that $A$ and $C_j$ intersect in at most one element.
    So
    \begin{align*}
        t \geq& |\{ j \ | \ A \cap C_j \neq \emptyset \}| = \sum_{a \in A} |\{ j \ | \ a \in C_j \}| = \sum_{i = 0}^n \ \sum_{a \in A \cap L_i} |\{ j \ | \ a \in C_j \}| \\
        =& \sum_{i = 0}^n \ \sum_{a \in A \cap L_i} k_i = \sum_{i = 0}^n k_i |A \cap L_i| = \sum_{i = 0}^n \frac {|A \cap L_i|} {|L_i|} t
    \end{align*}
    The claim follows by canceling $t$.

    \paragraph{(i) $\Rightarrow$ (ii)}
    Consider $A := F \cup (L_{i - 1} \setminus \partial F)$.
    This is clearly an antichain, as for $x \in F, y \in L_{i - 1} \setminus \partial F$ have $y \not< x$.
    However, $r(y) < r(x)$ so also $x \not< y$, thus $x$ and $y$ are incomparable.
    Clearly elements from the same layer are incomparable.

    Thus, the assumption yields that
    \begin{equation*}
        \frac {|L_{i - 1}| - |\partial F|} {|L_{i - 1}|} + \frac {|F|} {|L_i|} = \frac {|A \cap L_{i - 1}|} {|L_{i - 1}|} + \frac {|A \cap L_i|} {|L_i|} = \sum_{j = 0}^n \frac {|A \cap L_j|} {|L_j|} \leq 1
    \end{equation*}
    This gives
    \begin{equation*}
        1 + \frac {|F|} {|L_i|} \leq 1 + \frac {|\partial F|} {|L_{i - 1}|}
    \end{equation*}
    and the claim follows.
\end{proof}

\end{document}