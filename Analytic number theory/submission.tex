\documentclass{scrartcl}

\usepackage{graphicx}
\usepackage[utf8]{inputenc}
\usepackage[T1]{fontenc}
\usepackage[english]{babel}
\usepackage{amsmath}
\usepackage{mathtools}
\usepackage{amssymb}
\usepackage{listings}
\usepackage{amsthm}
\usepackage[style=english]{csquotes}
\usepackage[language=english, backend=biber, style=alphabetic, sorting=nyt]{biblatex}

\addbibresource{bibliography.bib}

\title{Miniproject - Analytic Number Theory}
\author{Simon Pohmann}
\date{}

\newcommand{\primes}{\mathbb{P}}
\newcommand{\R}{\mathbb{R}}
\newcommand{\N}{\mathbb{N}}
\newcommand{\Z}{\mathbb{Z}}
\newcommand{\Q}{\mathbb{Q}}
\newcommand{\C}{\mathbb{C}}
\newcommand{\divides}{\ | \ }
\newcommand{\units}{\times}
\newcommand{\sgn}{\mathrm{sgn}}
\newcommand\restr[2]{{
    \left.\kern-\nulldelimiterspace
    #1
    \vphantom{\big|}
    \right|_{#2}
}}

\theoremstyle{definition}
\newtheorem{definition}{Definition}
\newtheorem{lemma}[definition]{Lemma}
\newtheorem{remark}[definition]{Remark}
\newtheorem{proposition}[definition]{Proposition}
\newtheorem{example}[definition]{Example}
\newtheorem{corollary}[definition]{Corollary}

\begin{document}
\maketitle
We use the convention that $\N = \{ n \in \Z \ | \ n \geq 0 \}$.
Further, we write $a \divides b$ if $a$ divides $b$ and $a \perp b$ if $a$ and $b$ are coprime.
Finally, let $\primes$ be the set of prime numbers in $\N$.

\section{Part I}

For convenience, we include the definition of a Dirichlet character from the task description first.
\begin{definition}
    Let $q \geq 2$, then a \emph{Dirichlet character (mod $q$)} is a function $\chi: \N \to \C$ such that
    \begin{itemize}
        \item $\chi$ is completely multiplicative, so $\chi(a)\chi(b) = \chi(ab)$
        \item $\chi$ is periodic modulo $q$, so $\chi(n + q) = \chi(n)$
        \item $\chi(n) \neq 0$ if and only if $n \perp q$
    \end{itemize}
\end{definition}
First, we will give another characterization of Dirichlet characters.
\begin{lemma}[Characterization of Dirichlet characters]
    \label{prop:characterization_dirichlet_character}
    We have a one-to-one correspondence between Dirichlet characters mod $q$ and group homomorphisms $(\Z/q\Z)^\units \to \C^\units$ via
    \begin{align*}
        \{ \chi: (\Z/q\Z)^\units \to \C^\units \ | \ \chi \ \text{group hom} \} &\to \{ \chi: \N \to \C \ | \ \chi \ \text{Dirichlet character mod $q$} \} \\
        \chi &\mapsto \tilde{\chi} := \left( \N \to \C, \ n \mapsto \begin{cases}
            \chi([n]_q) & \text{if $n \perp q$} \\
            0 & \text{otherwise}
        \end{cases} \right)
    \end{align*}
\end{lemma}
\begin{proof}
    First of all, we show that the map is well-defined. Let $\chi: (\Z/q\Z)^\units \to \C^\units$ be a (multiplicative) group homomorphism, and we show that $\tilde{\chi}$ is a Dirichlet character.
    
    Note that property (ii) and (iii) directly follow from the definition, as $\tilde{\chi}(n)$ only depends on the value of $n \mod q$.
    So consider some $a, b \in \N$.
    If both $a \perp q$ and $b \perp q$ then
    \begin{equation*}
        \tilde{\chi}(a)\tilde{\chi}(b) = \chi([a])\chi([b]) = \chi([ab]) = \tilde{\chi}(ab)
    \end{equation*}
    as also $ab \perp q$.

    On the other hand, if $a \not\perp q$ or $b \not\perp q$ have $\chi(a) = 0$ resp. $\chi(b) = 0$.
    We also have in this case that $ab \not\perp q$ and so
    \begin{equation*}
        \chi(a)\chi(b) = 0 = \chi(ab)
    \end{equation*}
    Now it is left to show that the correspondence is a bijection.
    Clearly, if $\chi \neq \xi$ then $\chi(x) \neq \xi(x)$ for some $x \in (\Z/q\Z)^\units$ and so $\tilde{\chi}(n) \neq \tilde{\xi}(n)$ for some representative $n \in \N$ of $x$.
    
    To show surjectivity, consider some Dirichlet character $f: \N \to \C$ and construct a group homomorphism $\chi: (\Z/q\Z)^\units \to \C^\units$. 
    For each $x \in (\Z/q\Z)^\units$, there is a representative $n \in \N$ of $x$ and as $f(n)$ does not depend on the choice of $n$, we may define $\chi(x) := f(n)$.
    Note that as $x \in (\Z/q\Z)^\units$, we find $n \perp q$ and so $f(n) \neq 0$, i.e. $f(n) \in \C^*$.
    Then clearly for $a, b \in (\Z/q\Z)^*$ with representatives $n, m \in \N$ have
    \begin{equation*}
        \chi(ab) = f(nm) = f(n)f(m) = \chi(a)\chi(b)
    \end{equation*}
    So $\chi$ is a well-defined group homomorphism and we obviously have $\tilde{\chi} = f$.
\end{proof}
For simplicity of notation we sometimes will identify a Dirichlet character and its group homomorphism if it is always clear which one is meant.
\begin{example}[Ex (i)]
    \label{ex:nontrivial_dirichlet_character_mod_4}
    The function
    \begin{equation*}
        f: \N \to \C, \quad n \mapsto \begin{cases}
            0 & \text{if $n \equiv 0, 2 \mod 4$} \\
            1 & \text{if $n \equiv 1 \mod 4$} \\
            -1 & \text{if $n \equiv 3 \mod 4$}
        \end{cases}
    \end{equation*}
    is a Dirichlet character.
\end{example}
\begin{proof}
    This follows directly from Lemma~\ref{prop:characterization_dirichlet_character}, as $f = \tilde{\chi}$ for the group homomorphism
    \begin{equation*}
        \chi: (\Z/4\Z)^\units = \{ 1, 3 \} \to \C^*, \quad 1 \mapsto 1, \ 3 \mapsto -1
    \end{equation*}
    (this is a group homomorphism, as $3^2 = 9 \equiv 1 \mod 4$)
\end{proof}
Now we want to define Dirichlet series of Dirichlet characters.
\begin{proposition}
    For a Dirichlet character $\chi: \N \to \C$ and some $\epsilon > 0$, the series
    \begin{equation*}
        L(s, f) := \sum_{n \geq 1} f(n) n^{-s}
    \end{equation*}
    converges uniformly on $\Re(s) \geq 1 + \epsilon$.
    We will call it the Dirichlet series of $\chi$.
\end{proposition}
\begin{proof}
    By Lemma~\ref{prop:characterization_dirichlet_character}, we know that $\chi$ corresponds to a group homomorphism $\chi': (\Z/q\Z)^\units \to \C^\units$ such that $\chi(\N) = \chi((\Z/q\Z)^*) \cup \{ 0 \} \subseteq \C$ is a finite subset of $\C$.
    Hence, there is $C > 0$ with $|\chi(n)| \leq C$ for all $n \in \N$, and it follows that
    \begin{equation*}
        \sum_{1 \leq n \leq X} \left| f(n) n^{-s} \right| \leq \sum_{1 \leq n \leq X} C \left| n^{-s} \right| \leq C \sum_{1 \leq n \leq X} n^{-1 - \epsilon}
    \end{equation*}
    which is finite.
\end{proof}
\begin{proposition}
    \label{prop:dirichlet_series_at_one}
    Let $\chi: (\Z/q\Z)^\units \to \C^\units$ be a group homomorphism. Then for the associated Dirichlet character $\tilde{\chi}$ we have that
    \begin{equation*}
        \lim_{s \to 1^+} L(s, \tilde{\chi}) \ \text{exists} \ \Leftrightarrow \ \sum_{x \in (\Z/q\Z)^\units} \chi(x) = 0
    \end{equation*}
    In this case, have that
    \begin{equation*}
        \lim_{s \to 1^+} L(s, \tilde{\chi}) = \sum_{n \geq 1} f(n) n^{-s}
    \end{equation*}
    where the right sum converges (but not absolutely) for $\Re(s) > 0$.
\end{proposition}
\begin{proof}
    Let $c = \sum_{x \in (\Z/q\Z)^\units} \chi(x)$.
    For the direction $\Rightarrow$ assume that $c \neq 0$.
    Then have for $\Re(s) > 1$ that
    \begin{align*}
        \sgn(c) \sum_{n \geq 1} \tilde{\chi}(n) n^{-s} &= \sum_{n \geq 1} \ \sum_{0 \leq k < q} \sgn(c) \tilde{\chi}(qn + k) (qn + k)^{-s} \\
        &\geq \sum_{n \geq 1} \ \sum_{0 \leq k < 1}\sgn(c) \tilde{\chi}(qn + k) (qn + n)^{-s} \\
        &= \sum_{n \geq 1} \sgn(c) (qn + n)^{-s} \underbrace{\sum_{0 \leq k < q} \tilde{\chi}(qn + k)}_{= c} \\
        &\geq \frac {|c|} {(q + 1)^s} \sum_{n \geq 1} n^{-s} = \frac {|c|} {(q + 1)^s} \zeta(s)
    \end{align*}
    which clearly has a pole at $s = 1$. Hence $\lim_{s \to 1^+} L(s, \tilde{\chi})$ cannot exist.

    For the other direction, assume that $c = 0$.
    Again, have for $\Re(s) > 1$ that
    \begin{align*}
        \sum_{n \geq 1} \tilde{\chi}(n) n^{-s} &= \sum_{n \geq 1} \ \sum_{0 \leq k < q} \tilde{\chi}(qn + k) (qn + k)^{-s} \\
        &= \sum_{n \geq 1} \ \sum_{0 \leq k < q} \tilde{\chi}(qn + k) \Bigl( (qn)^{-s} + (qn + k)^{-s} - (qn)^{-s} \Bigr)
    \end{align*}
    Observe that by Bernoulli's inequality, have
    \begin{align*}
        (qn)^{-s} - (qn + k)^{-s} &= \frac {(qn)^s - (qn + k)^s} {(q^2n^2 + qnk)^s} = (qn)^{s} \frac {1 - (1 + k(qn)^{-1})^s} {(q^2n^2 + qnk)^s} \\
        &\leq (qn)^s \frac {sk(qn)^{-1}} {(q^2n^2 + qnk)^s} = \frac {sk} {qn(qn + k)^s} = O(sn^{-s - 1})
    \end{align*}
    As $\chi((\Z/q\Z)^\units) \subseteq \C$ is finite, find $C > 0$ with $|\tilde{\chi}(n)| \leq C$ for all $n \in \N$. Then
    \begin{align*}
        \sum_{n \geq X} \tilde{\chi}(n) n^{-s} &= O(qCX^{-s}) + \sum_{n \geq X/q} \ \sum_{0 \leq k < q} \tilde{\chi}(qn + k) \Bigl( (qn)^{-s} + O(sn^{-s - 1}) \Bigr) \\
        &= O(qCX^{-s}) + \sum_{n \leq X/q} \Bigl( (qn)^{-s} c + \sum_{0 \geq k < q} O(Csn^{-s - 1}) \Bigr) = \\
        &= O(qCX^{-s}) + 0 + O\Bigl( Cqs \sum_{n \geq X/q} n^{-s - 1} \Bigr) \\
        &\leq O(qCX^{-s}) + O\Bigl( Cqs\zeta(s + 1) \Bigr)
    \end{align*}
    which is well-defined and finite for $\Re(s) > 0$.
    Further, the expression converges uniformly (as a function in $s$ on a neighborhood of $1$) to $0$ as $X \to \infty$. 
    So
    \begin{equation*}
        \sum_{n < X} \tilde{\chi}(n) n^{-s} \quad \text{converges uniformly to} \quad \sum_{n \geq 1} \tilde{\chi}(n) n^{-s}
    \end{equation*}
    as $X \to \infty$ (on a neighborhood of $1$). 
    Thus the limit is continuous and a continuation of $L(s, \tilde{\chi})$ which is defined on $\Re(s) > 1$.
    From this it follows that $\lim_{s \to 1} L(s, \tilde{\chi})$ exists and is equal to $\sum_n \tilde{\chi}(n) n^{-s}$.
\end{proof}
Applied to our example, we find
\begin{example}[Ex (ii)]
    Let $f: \N \to \C$ be the Dirichlet character from Example~\ref{ex:nontrivial_dirichlet_character_mod_4} with corresponding group homomorphism $\chi: (\Z/4\Z)^\units \to \C$.
    Then
    \begin{equation*}
        \sum_{x \in (\Z/4\Z)^*} \chi(x) = \chi(1) + \chi(3) = 1 - 1 = 0
    \end{equation*}
    and so by Lemma~\ref{prop:dirichlet_series_at_one} the limit $\lim_{s \to 1^+} L(s, f)$ exists.
    The lemma further yields that
    \begin{align*}
        \lim_{s \to 1} L(s, f) &= \sum_{n \geq 1} f(n) n^{-1} = \sum_{n \geq 0} \frac {f(4n + 1)} {4n + 1} + \frac {f(4n + 3)} {4n + 3} = \sum_{n \geq 0} \frac 1 {4n + 1} - \frac 1 {4n + 3} \\
        &= 2 \sum_{n \geq 0} \frac 1 {(4n + 1)(4n + 3)} > 0
    \end{align*}
    is positive. 
    Wolfram Alpha \cite{wolfram_alpha} can give an explicit value to this sum, using the digamma function $\psi$. Namely
    \begin{equation*}
        \sum_{x \in (\Z/4\Z)^\units} f(n) n^{-1} = \frac 1 4 (\psi(\frac 7 4) - \psi(\frac 5 4))
    \end{equation*}
    which seems to be $\frac 1 4$.
\end{example}
Now we want to study the series
\begin{equation*}
    \sum_p f(p) p^{-s}
\end{equation*}
For this, we are first interested in how many primes $\equiv 1, 3 \mod 4$ there are.
\begin{lemma}
    \label{prop:prime_factor_3_mod_4}
    Let $n \equiv 3 \mod 4$. Then $n$ has a prime factor $p \equiv 3 \mod 4$.
\end{lemma}
\begin{proof}
    Use induction on $n$.
    If $n = 3$, the claim is trivial. 
    So let $n > 3$. If $n$ is prime, the claim again follows.
    Otherwise, have $n = ab$ with nontrivial divisors $a, b$.
    However, $3 \equiv n$ is not a square modulo $4$, so find that $a \not\equiv b \mod 4$.
    As both $a$ and $b$ must be odd, we see that either $a \equiv 3 \mod 4$ or $b \equiv 3 \mod 4$ and the claim follows by the induction hypothesis.
\end{proof}
\begin{corollary}[Ex (iii)]
    There are infinitely many primes $p$ with $p \equiv 3 \mod 4$.
\end{corollary}
\begin{proof}
    Assume there were only finitely many, say $p_1, ..., p_N$.
    Let $P := p_1 ... p_N$ if $N$ is even and $P := p_1^2 p_2 ... p_N$ if $N$ is odd.
    Then
    \begin{equation*}
        P \equiv 3^{2 \lceil \frac N 2 \rceil} \equiv 1^{\lceil \frac N 2 \rceil} = 1 \mod 4
    \end{equation*}
    Thus, by Lemma~\ref{prop:prime_factor_3_mod_4}, $P + 2$ has a prime factor $q \equiv 3 \mod 4$.
    However, $q \neq p_i$ as $p_i \perp P + 2$ for all $i$ (if $p_i \divides P + 2$, then $p_i \divides P + 2 - P = 2$, a contradiction).
    This contradicts our assumption.
\end{proof}
For the case of primes $\equiv 1 \mod 4$, I have remembered the two-square theorem and its connection to primes in the ring $\Z[i]$ of Gaussian integers, and somehow my train of thoughts went into Algebraic Number Theory.
After some research, I have found an exercise in \cite[Chapter I, §10]{neukirch} that requires the reader to prove the following proposition.
\begin{proposition}
    \label{prop:primes_1_mod_q}
    Let $q \geq 3$ be an integer.
    Then there are infinitely many primes $p$ with $p \equiv 1 \mod q$.
\end{proposition}
\begin{proof}
    Assume there were only finitely many such primes $p_i$, then we have their product $P = \prod_i p_i \in \Z$.
    Consider now the $q$-th cyclotomic polynomial $\Phi_q$.
    Clearly $\Phi_q(qPX) - 1 \in \Q[X]$ has at most $\phi(q)$ zeros, so there exists some $x \in \Z$ with $\Phi_q(qPx) \neq 1$ (this ``Ansatz'' was given as a hint).

    Let now $K = \Q(\omega_q)$ be the $q$-th cyclotomic number field with a primitive $q$-th root of unity $\omega_q$ (i.e. $\Phi_q(\omega_q) = 0$).
    Let further $\mathcal{O} \subseteq K$ be the ring of integral elements over $\Z$ in $K$.
    The prime decomposition law for Dedekind ring extension \cite[Chapter I, Prop 8.3]{neukirch} tells us that for a prime $p$, the ideal $(p)$ is reducible in $\mathcal{O}$ if and only if $\Phi_q \mod p$ is reducible.
    As $(\Z/p\Z)^\units$ is cyclic of order $p - 1$, this is the case if and only if $q \divides p - 1$, i.e. $p \equiv 1 \mod q$.

    Now consider the element $\alpha = \omega_q - xqP \in \mathcal{O}$. Then
    \begin{align*}
        N_{K/\Q}(\alpha) &= \prod_{\sigma: K \to \C \ \text{$\Q$-field homomorphism}} \sigma(\omega_q - xqP) \\
        &= \prod_{\sigma} (\sigma(\omega_q) - xqP) = \mathrm{MiPo}_\Q(\omega_q)(xqP) = \Phi_q(xqP) \neq 1
    \end{align*}
    as $\mathrm{MiPo}_\Q(\omega_q) = \prod_\sigma (\sigma(\omega_q) - X)$.
    Hence, $\alpha$ is not a unit in $\mathcal{O}$.
    On the other hand, $(\alpha)$ is coprime to $(p_i)$ for each $p_i$, as 
    \begin{equation*}
        \omega_q = \alpha - xqP \in (\alpha) + (p_i) \quad \text{and} \quad \omega_q \in \mathcal{O}^\units
    \end{equation*}
    By our assumption, the only prime ideals in $\mathcal{O}$ are the prime ideal factors of $(p_i)$ and $(p)$ for $p \neq p_i$.
    Thus, the prime ideal factorization of $(\alpha)$ consists only of prime ideals $(p), p \neq p_i$ and it follows that $(\alpha) = (n)$ for some integer $n \geq 2$.
    As $\omega_q$ and $xqP \in \Z$ are $\Q$-linearly independent, we see that $n \divides \omega_q$ and $n \divides xqP$.
    However, the former is a contradiction, as $\omega_q \in \mathcal{O}^\units$ is a unit and no $n \geq 2$ is a unit.
\end{proof}
The book also mentions that the general case can be proven by using L-series in algebraic number fields.
\begin{corollary}[Ex (iii)]
    There are infinitely many primes $p$ with $p \equiv 1 \mod 4$.
\end{corollary}
\begin{proof}
    This is just a special case of Prop.~\ref{prop:primes_1_mod_q}.
\end{proof}
\begin{example}[Ex (iii)]
    Using a computer, we can also study the actual frequency of prime numbers $\equiv 1, 3 \mod 4$ among e.g. the first $10^8$ integers.
    This seems to indicate that both numbers are asymptotically equal.
    For example, there are 332180 primes $\equiv 1 \mod 4$ and 332398 primes $\equiv 3 \mod 4$ smaller than $10^8$.
    To find these numbers, the following python code was used.
    \lstinputlisting[language = python]{./primes.py}
\end{example}

\section{Part II}
We have already shown that Dirichlet characters are, in principle, group homomorphisms $(\Z/q\Z)^\units \to \C^\units$.
If we now assume $q$ to be prime, we get an even nicer characterization.
\begin{corollary}[Ex (i)]
    \label{prop:dirichlet_character_image_group}
    Let $\chi, \chi': \N \to \C$ be Dirichlet characters mod $q$ and $r$ a primitive root modulo $q$.
    If $\chi(r) = \chi'(r)$, then $\chi = \chi'$.
    Further, have that $\chi(n)^{q - 1} = 1$ for all $n \in \N$ with $n \perp q$.
\end{corollary}
\begin{proof}
    The properties follow directly from Lemma~\ref{prop:characterization_dirichlet_character}.
    Let $f, f': (\Z/q\Z)^\units \to \C^\units$ be the associated group homomorphisms of $\chi, \chi'$ as in Lemma~\ref{prop:characterization_dirichlet_character}.
    If $f(r) = \chi(r) = \chi'(r) = f'(r)$ then clearly $f = f'$, as these are group homomorphisms and $\langle r \rangle = (\Z/q\Z)^\units$.
    Hence $\chi = \chi'$.

    Further, have for $n \in \N$ with $n \perp q$ that $n \in (\Z/q\Z)^\units$ and thus
    \begin{equation*}
        n^{q - 1} = n^{\phi(q)} = n^{|(\Z/q\Z)^\units|} = 1 
    \end{equation*}
    As $f$ is a group homomorphism, find
    \begin{equation*}
        \chi(n)^{q - 1} = f(n)^{q - 1} = f(n^{q - 1}) = f(1) = 1
    \end{equation*}
\end{proof}
This correspondence also works in the other direction.
\begin{corollary}[Ex (ii)]
    \label{prop:dirichlet_character_primitive_root}
    Let $\omega \in \C$ be a $(q - 1)$-th root of unity, i.e. $\omega^{q - 1} = 1$ and let $r \in (\Z/q\Z)^\units$ be a primitive root.
    Then
    \begin{equation*}
        g: \N \to \C, \quad n \mapsto \begin{cases}
            \omega^{\log_r n} & \text{if $n \perp q$} \\
            0 & \text{otherwise}
        \end{cases}
    \end{equation*}
    is a well-defined Dirichlet character.
\end{corollary}
\begin{proof}
    Follows again directly from Lemma~\ref{prop:characterization_dirichlet_character}, as $r \mapsto \omega$ induces a unique group homomorphism $(\Z/q\Z)^\units \to \C^\units$.
    The associated Dirichlet character is obviously $g$.
\end{proof}
Note that the image of a group homomorphism $\chi: (\Z/q\Z)^\units \to \C^\units$ is a subgroup of $\C^\units$.
Using Corollary~\ref{prop:dirichlet_character_image_group}, we can describe it quite concretely.
\begin{proposition}
    Let $\chi: \N \to \C$ be a Dirichlet character with group homomorphism $f: (\Z/q\Z)^\units \to \C^\units$.
    Then $\mathrm{im}f \leq S$ is a subgroup where $S = \{ \omega_q^k \ | \ k \in \Z \}$ is the group of $q$-th roots of unity.
    
    It is a fact from Algebra that $S \cong (\Z/q\Z)^\units$, hence Dirichlet characters modulo a prime $q$ are in 1-to-1 correspondence with the endomorphism monoid $\mathrm{End}((\Z/q\Z)^\units)$.
\end{proposition}
\begin{proof}
    We have that $S = \{ x \in \C^\units \ | \ x^{q - 1} = 1 \}$ and the claim directly follows from Corollary~\ref{prop:dirichlet_character_image_group}.
\end{proof}
Note that the endomorphism monoid $\mathrm{End}((\Z/q\Z)^\units)$ is not a group, except in the trivial case $q = 2$.
The reason is that e.g. the trivial group homomorphism $r \mapsto 1$ is not surjective and thus not invertible.

By Corollary~\ref{prop:dirichlet_character_primitive_root} each group endomorphism $f \in \mathrm{End}((\Z/q\Z)^\units)$ is determined by its value at a primitive root of unity $r \in (\Z/q\Z)^\units$, hence
\begin{equation*}
    |\mathrm{End}((\Z/q\Z)^\units)| = |(\Z/q\Z)^\units| = q - 1
\end{equation*}
It follows that there are exactly $q - 1$ distinct Dirichlet characters modulo a prime $q$.
\begin{remark}
    It is again a fact that $(\Z/p^k\Z)^\units$ is cyclic for an odd prime $p$ and $k \geq 1$.
    Hence, everything up to now can also be done for odd prime powers, if we replace $q - 1$ by $\phi(q)$.
\end{remark}
Because of Lemma~\ref{prop:dirichlet_series_at_one} it might seem like a good idea to study in which cases the value $\sum_{x \in (\Z/q\Z)^\units} \chi(x)$ is zero.
\begin{proposition}[Ex (iii)]
    Let $\chi_0$ be the trivial Dirichlet character given by $r \mapsto 1$.
    Then
    \begin{align*}
        \sum_{a \in (\Z/q\Z)^\units} \chi(a) &= \begin{cases}
            q - 1 & \text{if $\chi = \chi_0$} \\
            0 & \text{otherwise}
        \end{cases}, \\
        \sum_{\chi \in \mathrm{End}((\Z/q\Z)^\units)} \chi(a) &= \begin{cases}
            q - 1 & \text{if $a \equiv 1 \mod q$} \\
            0 & \text{otherwise}
        \end{cases}
    \end{align*}
    Furthermore, for $b \perp q$ have
    \begin{equation*}
        \sum_{\chi \in \mathrm{End}((\Z/q\Z)^\units)} \chi(a)\overline{\chi(b)} = \begin{cases}
            q - 1 & \text{if $a \equiv b \mod q$} \\
            0 & \text{otherwise}
        \end{cases}
    \end{equation*}
\end{proposition}
\begin{proof}
    Clearly
    \begin{equation*}
        \sum_{a \in (\Z/q\Z)^\units} \chi_0(a) = q - 1 \quad \text{and} \quad \sum_{\chi \in \mathrm{End}((\Z/q\Z)^\units)} \chi(1) = \sum_{\chi \in \mathrm{End}((\Z/q\Z)^\units)} 1 = q - 1
    \end{equation*}
    So it is left to show that we get zero in the other cases.
    
    Consider a Dirichlet character $\chi \neq \chi_0$ given by $r \mapsto \xi$ for a $q$-th root of unity $\xi \neq 1$.
    Then
    \begin{equation*}
        \sum_{a \in (\Z/q\Z)^\units} \chi(a) = \sum_{k = 0}^{q - 2} \chi(r^k) = \sum_{k = 0}^{q - 2} \xi^k = \frac {1 - \xi^{q - 1}} {q - \xi} = 0 
    \end{equation*}
    By using the earlier results on the structure of $\mathrm{End}((\Z/q\Z)^\units)$ we see that for $a = r^k \not\equiv 1 \mod q$, have
    \begin{align*}
        \sum_{\chi \in \mathrm{End}((\Z/q\Z)^\units)} \chi(a) &= \sum_{\chi \in \mathrm{End}((\Z/q\Z)^\units)} \chi(r)^k \\
        &= \sum_{\xi \ \text{$q$-th root of unity}} \xi^k = \sum_{l = 0}^{q - 2} \omega^{kl} = \frac {1 - (\omega^{q - 1})^k} {1 - \omega^k} = 0 
    \end{align*}
    where $\omega$ is a primitive $q$-th root of unity.
    
    For the last part, note that for any $q$-th root of unity $\xi$, we have $\xi\overline{\xi} \in \R$ with $\xi\overline{\xi} = |\xi|^2 > 0$.
    Furthermore, $\overline{\xi}$ is also a $q$-th root of unity, and so we see that $\xi\overline{\xi} = 1$.
    It follows that for any Dirichlet character $\chi$ have $\overline{\chi}(a) = \chi(a^{-1})$ (where the inversion happens in $(\Z/q\Z)^\units$).
    Thus
    \begin{equation*}
        \sum_{\chi \in \mathrm{End}((\Z/q\Z)^\units)} \chi(a)\overline{\chi}(b) = \sum_{\chi \in \mathrm{End}((\Z/q\Z)^\units)} \chi(ab^{-1}) = \begin{cases}
            q - 1 & \text{if $ab^{-1} = 1 \in (\Z/q\Z)^\units$} \\
            0 & \text{otherwise}
        \end{cases}
    \end{equation*}
    The condition $ab^{-1} = 1$ is equivalent to $a \equiv b \mod q$, so the claim follows.
\end{proof}

\printbibliography
\end{document}