\documentclass{scrartcl}

\usepackage{graphicx}
\usepackage[utf8]{inputenc}
\usepackage[T1]{fontenc}
\usepackage[english]{babel}
\usepackage{amsmath}
\usepackage{mathtools}
\usepackage{amssymb}
\usepackage{amsthm}
\usepackage{listings}
\usepackage[style=english]{csquotes}
\usepackage[language=english, backend=biber, style=alphabetic, sorting=nyt]{biblatex}

\addbibresource{bibliography.bib}

\title{Miniproject - Algebraic Geometry}
\author{Simon Pohmann}
\date{}

\newcommand{\R}{\mathbb{R}}
\newcommand{\N}{\mathbb{N}}
\newcommand{\Z}{\mathbb{Z}}
\newcommand{\Q}{\mathbb{Q}}
\newcommand{\C}{\mathbb{C}}
\newcommand{\I}{\mathbb{I}}
\newcommand{\V}{\mathbb{V}}
\newcommand{\Proj}{\mathbb{P}}
\newcommand{\Aff}{\mathbb{A}}
\newcommand{\GL}{\mathrm{GL}}
\newcommand{\Gr}{\mathrm{Gr}}
\newcommand{\sgn}{\mathrm{sgn}}
\newcommand{\extpow}{\mathchoice{{\textstyle\bigwedge}}
    {{\bigwedge}}
    {{\textstyle\wedge}}
    {{\scriptstyle\wedge}}}
\newcommand{\vspan}{\mathrm{span}}
\newcommand{\divides}{\ | \ }
\newcommand\restr[2]{{
    \left.\kern-\nulldelimiterspace
    #1
    \vphantom{\big|}
    \right|_{#2}
}}

\newtheorem{definition}{Definition}
\newtheorem{lemma}[definition]{Lemma}
\newtheorem{example}[definition]{Example}

\begin{document}
\maketitle

\begin{definition}
    Let $V$ be a vector space.
    Then define the $d$-th exterior power as
    \begin{equation*}
        \extpow^d(V) := V^{\otimes d} \ / \ \sum_{i = 1}^{d - 1} V^{\otimes (i - 1)} \otimes \bigl\{ v \otimes v' + v' \otimes v \ \bigm| \ v, v' \in V \bigr\} \otimes V^{\otimes (d - i - 1)}
    \end{equation*}
    Use the notation $v_1 \wedge ... \wedge v_d := [v_1 \otimes ... \otimes v_d] \in \extpow^k(V)$.
\end{definition}

\begin{lemma}
    \label{prop:basic_properties_exterior_product}
    Let $v_1, ..., v_d \in V$. Have for $\pi \in S_d$ that
    \begin{equation*}
        v_{\pi(1)} \wedge ... \wedge v_{\pi(k)} = \sgn(\pi) (v_1 \wedge ... \wedge v_d)
    \end{equation*}
    Furthermore if $v_i = v_j$ for some $i \neq j$, then
    \begin{equation*}
        v_1 \wedge ... \wedge v_d = 0
    \end{equation*}
\end{lemma}
\begin{proof}
    Note that
    \begin{equation*}
        u \otimes v \otimes v' \otimes w = -(u \otimes v' \otimes v \otimes w)
    \end{equation*}
    for all $u \in \extpow^{i - 1}(V), w \in \extpow^{(d - i - 1)}(V), v, v' \in V$.

    Every $\pi \in S_d$ has a decomposition $\pi = \xi_1 ... \xi_n$ into transpositions $\xi_i$.
    Applying this inductively, we find that
    \begin{equation*}
        v_1 \wedge ... \wedge v_d = \sgn(\xi_i ... \xi_n) (v_{(\xi_i ... \xi_n)(1)} \wedge ... v_{(\xi_i ... \xi_n)(k)})
    \end{equation*}
    and so
    \begin{equation*}
        v_1 \wedge ... \wedge v_d = \sgn(\pi) (v_{\pi(1)} \wedge ... \wedge v_{\pi(k)})
    \end{equation*}

    Furthermore, we find that
    \begin{equation*}
        u \otimes v \otimes v \otimes w = -(u \otimes v \otimes v \otimes w) = 0
    \end{equation*}
    must be zero.
    Hence, if $v_1, ..., v_d \in V$ with $v_i = v_j$ for some $i \neq j$, then there is a permutation $\pi \in S_d$ with $\pi(1) = i, \pi(2) = j$ and
    \begin{equation*}
        v_1 \wedge ... \wedge v_d = (\sgn(\pi))(v_i \wedge v_j \wedge v_{\pi(3)} \wedge ... \wedge v_{\pi(k)}) = \sgn(\pi) 0 = 0
    \end{equation*}
\end{proof}

\begin{lemma}[1a]
    Let $\dim(V) \leq 3$. Then every element of $\extpow^k(V)$ is decomposable.
\end{lemma}
\begin{proof}
    Now let $v_1, v_2, v_3$ be a set of generators of $V$.
    Consider $u_1 = \sum \lambda_i v_i, u_2 = \sum_i \mu_i v_i, u_3 = \sum_i \rho_i v_i$.
    Then by applying Lemma~\ref{prop:basic_properties_exterior_product}, we see that 
    \begin{align*}
        u_1 \wedge u_2 =& \sum_{i, j} \ \lambda_i \mu_j (\underbrace{v_i \wedge v_j}_{\mathclap{\text{$= 0$ if $i = j$}}}) = \sum_{i < j} \lambda_i \mu_j (v_i \wedge v_j) - \sum_{i > j} \lambda_i \mu_j (v_i \wedge v_j) \\
        =& \sum_{i < j} (\lambda_i \mu_j - \lambda_j \mu_i)(v_i \wedge v_j) = \alpha (v_1 \wedge v_2) + \beta (v_1 \wedge v_3) + \gamma (v_2 \wedge v_3) \\
        =& \begin{cases}
            \beta v_1 + \gamma v_2 \wedge \frac \alpha \beta v_2 + v_3 & \text{if $\beta \neq 0$} \\
            \alpha v_1 - \gamma v_3 \wedge v_2 & \text{otherwise}
        \end{cases}
    \end{align*}
    and
    \begin{align*}
        u_1 \wedge u_2 \wedge u_3 =& \sum_{i, j, l} \ \lambda_i \mu_j \rho_l (\underbrace{v_i \wedge v_j \wedge v_l}_{\mathclap{\text{$= 0$ unless $i, j, l$ pairwise distinct}}}) \\
        =& \sum_{\pi \in S_3} \lambda_{\pi(1)} \mu_{\pi(2)} \rho_{\pi(3)} (v_{\pi(1)} \wedge v_{\pi(2)} \wedge v_{\pi(3)}) \\
        =& \sum_{\pi \in S_3} \lambda_{\pi(1)} \mu_{\pi(2)} \rho_{\pi(3)} \sgn(\pi) (v_1 \wedge v_2 \wedge v_3) \\
        =& (v_1 \wedge v_2 \wedge v_3) \sum_{\pi \in S_3} \lambda_{\pi(1)} \mu_{\pi(2)} \rho_{\pi(3)} \sgn(\pi)
    \end{align*}
    are decomposable.
    Further, it is easy to see from Lemma~\ref{prop:basic_properties_exterior_product} that $\extpow^k(V) = \{ 0 \}$ for $k \geq 4$, which is trivially decomposable.
\end{proof}

\begin{example}[1b]
    Consider $V = k^4$. 
    Then the element $w := (e_1 \wedge e_2) + (e_3 \wedge e_4) \in \extpow^2(V)$ is not decomposable. 
\end{example}
\begin{proof}
    Assume it was, then there are $a, b \in k^4$ such that
    \begin{equation*}
        w = \sum_i a_i e_i \wedge \sum_j b_j e_j = \sum_{i < j} (a_i b_j - a_j b_i) (e_i \wedge e_j)
    \end{equation*}
    In other words
    \begin{equation*}
        a_1b_2 - a_2b_1 = 1, \ a_3b_4 - a_4b_3 = 1, \ a_i b_j - a_j b_i = 0 \ \text{for all $(i, j) \neq (1, 2), (3, 4)$}
    \end{equation*}
    Clearly $a_1 b_2 \neq 0$ or $a_2 b_1 \neq 0$.
    Similarly, have $a_3 b_4 \neq 0$ or $a_4 b_3 \neq 0$.
    As all expressions are symmetric w.r.t swapping $a_1, b_2$ with $a_2, b_1$ and $a_3, b_4$ with $a_4, b_3$, we may assume wlog that $a_1 b_2, a_3 b_4 \neq 0$.

    Have $a_1 b_4 = a_4 b_1$ and $a_2 b_4 = a_4 b_2$.
    We know that $a_1 b_4 \neq 0$ and so
    \begin{equation*}
        \frac {a_2} {a_1} = \frac {a_2 b_4} {a_1 b_4} = \frac {a_4 b_2} {a_4 b_1} = \frac {b_2} {b_1} \ \Rightarrow \ a_2 b_1 = a_1 b_2
    \end{equation*}
    This contradicts $a_1 b_2 - a_2 b_1 = 1$.
\end{proof}

\begin{lemma}[1c]
    Let $d$ be even.
    An element $\omega \in \extpow^d V$ is decomposable if and only if $\omega \wedge \omega \in \extpow^{2d} V$ is zero.
\end{lemma}
\begin{proof}
    The direction $\Rightarrow$ even holds generally. Assume $\omega = v_1 \wedge ... \wedge v_d$.
    Then
    \begin{equation*}
        \omega \wedge \omega = v_1 \wedge ... \wedge v_d \wedge v_1 \wedge ... \wedge v_d = 0
    \end{equation*}
    by Lemma~\ref{prop:basic_properties_exterior_product}. 
    The other direction is more interesting.

    Let $\omega = v_1 + ... + v_t$ for linearly independent decomposable vectors $v_i \in \extpow^2 V$.
    Then
    \begin{align*}
        0 =& \omega \wedge \omega = \sum_{i, j} v_i \wedge v_j = \sum_{i < j} (v_i \wedge v_j) + (v_j \wedge v_i) \\
        =& \sum_{i < j} 2(v_i \wedge v_j) = 2\sum_i v_i \wedge \Bigl( \sum_{j > i} v_j \Bigr)
    \end{align*}
    Here we used that the permutation $\bigl(1 \ 2d\bigr)\bigl(2 \ (2d - 1)\bigr)...\bigl(d \ (d + 1)\bigr) \in S_{2d}$ has always sign $1$ (since $d$ is even).

    Note that for any nonzero decomposable vector
    \begin{equation*}
        u_1 \wedge u_2 \in \Bigl( \extpow^2 \vspan\{v_2, ..., v_t\} \Bigr) \setminus \{0\}
    \end{equation*}
    find
    \begin{equation*}
        u_1, u_2 \in \vspan\{v_2, ..., v_t\}
    \end{equation*}
    In particular, we know that
    \begin{equation*}
        v_1 \wedge \Bigl( \sum_{j > i} v_j \Bigr) \in \extpow^2 \vspan\{v_2, ..., v_t\}
    \end{equation*}
    and so $v_1 \in \vspan\{v_2, ..., v_t\}$ unless $\sum_{j > i} v_j = 0$.
    We assumed that the $v_i$ are linearly independent, so the former would give a contradiction.
    Hence $\sum_{j > i} v_j = 0$ and thus $t = 1$, i.e. $\omega = v_1$ is decomposable.
\end{proof}

\begin{lemma}[2a]
    \label{prop:linear_transform_extpow}
    Let $A = (a_{ij}) \in \GL_d(k)$ and $v_1, ..., v_d \in V$.
    Then
    \begin{equation*}
        \Bigl( \sum_j a_{1j}v_j \Bigl) \wedge ... \wedge \Bigl( \sum_{j} a_{dj} v_j \Bigr) = \det(A) (v_1 \wedge ... \wedge v_d)
    \end{equation*}
\end{lemma}
\begin{proof}
    By a direct computation using Lemma~\ref{prop:basic_properties_exterior_product}, we find
    \begin{align*}
        &\Bigl( \sum_j a_{ij} v_j \Bigr) \wedge ... \wedge \Bigl( \sum_j a_{dj} v_j \Bigr) = \sum_{j_1, ..., j_d} a_{1j_1} ... a_{dj_d} (v_{j_1} \wedge ... \wedge v_{j_d}) \\
        = &\sum_{\pi \in S_d} a_{1\pi(1)} ... a_{d\pi(d)} (v_{\pi(1)} \wedge ... \wedge v_{\pi(d)}) \\
        = &\sum_{\pi \in S_d} a_{1\pi(1)} ... a_{d\pi(d)} \sgn(\pi) (v_1 \wedge ... \wedge v_d) \\
        = &(v_1 \wedge ... \wedge v_d) \ \sum_{\pi \in S_d} \sgn(\pi) \prod_{j = 1}^d a_{j\pi(j)} = \det(A) (v_1 \wedge ... \wedge v_d)
    \end{align*}
    where the last equality holds due to the Leibniz determinant formula.
\end{proof}
We see that if there are two bases $v_1, ..., v_d$ and $u_1, ..., u_d$ of a $d$-dimensional vector space $U$, then there exists a basis change matrix $A = (a_{ij}) \in \GL_d(k)$ with
\begin{equation*}
    u_i = \sum_j a_{ij} v_j
\end{equation*}
So by the Lemma, it follows that
\begin{equation*}
    u_1 \wedge ... \wedge u_d = \det(A) (v_1 \wedge ... \wedge v_d)
\end{equation*}
As $v_1, ..., v_d$ resp. $u_1, ..., u_d$ are bases, they are linearly independent and in particular, we see that
\begin{equation*}
    v_1 \wedge ... \wedge v_d \neq 0 \quad \text{and} \quad u_1 \wedge ... \wedge u_d \neq 0
\end{equation*}
Hence they have well-defined images $[v_1 \wedge ... \wedge v_d]$ resp. $[u_1 \wedge ... \wedge u_d]$ in the projective space $\Proj(\extpow^d V)$.
By the above, find
\begin{equation*}
    [v_1 \wedge ... \wedge v_d] = [u_1 \wedge ... \wedge u_d]
\end{equation*}
This allows us to study the Grassmanian $\Gr(d, V)$ of a fixed vector space $V$.
\begin{definition}
    Define the map
    \begin{equation*}
        \phi: \Gr(d, V) \to \Proj(\extpow^d V), \quad \vspan\{v_1, ..., v_d\} \mapsto [v_1 \wedge ... \wedge v_d]
    \end{equation*}
    which is well-defined by Lemma~\ref{prop:linear_transform_extpow} as described above.
\end{definition}
\begin{lemma}[1a]
    We have
    \begin{equation*}
        \mathrm{im}\phi = D := \{ [v] \in \Proj(\extpow^d V) \ | \ \text{$v$ decomposable}\}
    \end{equation*}
\end{lemma}
\begin{proof}
    First of all, note that the set $D$ is well-defined, as $v$ is decomposable if and only if $\lambda v$ is decomposable, for all $\lambda \in k^*$.

    By definition of $\phi$, we can directly observe that $\mathrm{im}\phi \subseteq D$.
    So consider an element $[v] \in D$.
    As $v$ is decomposable, it follows that $v = v_1 \wedge ... \wedge v_d$ for $v_i \in V$.
    Not it suffices to show that the $v_i$ are linearly independent, then clearly $\vspan\{v_1, ..., v_d\}$ is a well-defined $d$-dimensional vector subspace of $V$, thus an element of $\Gr(d, V)$.

    Assume not, then there is a nonzero vector $a_1 \in k^d$ with $\sum a_{1i} v_i = 0$.
    Clearly, we can extend $a_1$ to a basis $a_1, ..., a_d$ of $k^d$, which gives a matrix $A = (a_{ij}) \in \GL_d(k)$.

    However by Lemma~\ref{prop:linear_transform_extpow} we now get
    \begin{align*}
        0 =& 0 \wedge \Bigl( \sum_j a_{2j} v_j \Bigr) \wedge ... \wedge \Bigl( \sum_j a_{dj} v_j \Bigr) = \Bigl( \sum_j a_{1j} v_j \Bigr) \wedge ... \wedge \Bigl( \sum_j a_{dj} v_j \Bigr) \\
        =& \det(A) (v_1 \wedge ... \wedge v_d)
    \end{align*}
    and so $v = v_1 \wedge ... \wedge v_d = 0$ as $\det(A) \neq 0$.
    However, $v$ was a representative of a point in $\Proj(\extpow^d V)$, a contradiction.
    The claim follows.
\end{proof}
\end{document}  